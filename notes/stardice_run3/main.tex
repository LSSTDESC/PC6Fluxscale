% 
% ======================================================================
\RequirePackage{docswitch}
% \flag is set by the user, through the makefile:
%    make note
%    make apj
% etc.
\setjournal{\flag}

\documentclass[\docopts]{\docclass}

% You could also define the document class directly
%\documentclass[]{emulateapj}

% Custom commands from LSST DESC, see texmf/styles/lsstdesc_macros.sty
\usepackage{lsstdesc_macros}

\usepackage{graphicx}
\graphicspath{{./}{./figures/}}
\bibliographystyle{apj}

% Add your own macros here:



% 
% ======================================================================

\begin{document}

\title{ StarDICE run 3: simulation and performance forecasts }

\maketitlepre

\begin{abstract}
  We describe a model of the expected StarDICE data. We use the model
  to produce simulated datasets. We then describe the current status
  of the reduction and analysis chain and process simulated data
  through this chain. This draws a forecast of the performance of the
  next campaign as a function of the number of nights and variability
  of atmospheric parameters.
\end{abstract}

% Keywords are ignored in the LSST DESC Note style:
\dockeys{photometry: calibration}

\maketitlepost

% ----------------------------------------------------------------------
% 

\section{Introduction}
\label{sec:intro}

For an isolated point source:
\begin{equation}
  \begin{split}
    \phi_i & \sim \mathcal{P}\left(\tau_\text{exp} \int_{\lambda} d\lambda R(i, \lambda, t) \left[B_i(\lambda) + A(\lambda, t) S(\lambda)\int_{p_i} dx \psi(x - x_0)\right] \right)\\
    \Phi & = \sum_{i \in {Aperture}} w_i \frac1{G(t)} \phi_i + n_i  \quad \text{with } n_i = \mathcal{N}(b_i(t), \sigma_r^2) 
\end{split}
\end{equation}


% ----------------------------------------------------------------------
\section{Components of the photometric response model}
\label{sec:model}

\subsection{Instrumental response}
\label{sec:instrument}

\subsubsection{Detector quantum efficiency }
\label{sec:qe}
François

\subsubsection{Filter transmission}
\label{sec:filters}
François

\subsubsection{Transmission of the optics}
\label{sec:optics}
Bertrand

\subsubsection{Instrumental PSF}
\label{sec:PSF}
\begin{itemize}
\item Définition Surfaces
\item Optique polynomiale / zeemacs
\end{itemize}

Nicolas

\subsection{Atmospherique transmission}
\label{sec:atmosphere}

\subsubsection{Horizontal line of sight}
\label{sec:hline}

Sylvie et Séb

\subsubsection{Stellar line of sight}
\label{sec:av}

Sylvie et Séb

\section{Light sources}
\label{sec:sources}

\subsection{LEDs}
\label{sec:leds}

François

\subsection{Stars}
\label{sec:stars}

Marc

\subsection{Background light}
\label{sec:background}

Sylvie, Bertrand, Séb

\section{Observation plan}
\label{sec:plan}

\subsection{Dataset for the pointing model}
\label{sec:pointingmodel}

\subsection{Dataset for the focus model}
\label{sec:focusmodel}

\subsection{Flatfields}
\label{sec:flatfields}

\subsection{Telescopic PSF studies}


\subsection{Science dataset}
\label{sec:data}

\section{Data reduction chain}
\label{sec:dataanalysis}

Marc et François
\subsection{Detrending}
\label{sec:detrending}

\subsection{image processing}
\label{sec:processing}

\subsubsection{Source detection}
\label{sec:detection}

\subsubsection{Centroiding}
\label{sec:centroiding}

\subsubsection{photometry}
\label{sec:photometry}

\subsubsection{astrometry}
\label{sec:astrometry}

\subsection{Ancillary data}
\label{sec:ancillary}

Séb

\subsection{Catalog analysis}
\label{sec:analysis}


\section{Performance forecast}
\label{sec:forecast}


\section{Conclusion}
\label{sec:conclusion}


\section{Commands}
\label{sec:commands}

There are a number of useful \LaTeX\xspace commands predefined in \code{macros.tex}.
Notice that the section labels are prefixed with \code{sec:} to allow the use of the \verb=\secref= command to reference a section (\ie, \secref{intro}).
Figures can be referenced with the \verb=\figref= command, which assumes that the figure label is prefixed with \code{fig:}.
In \figref{example} we show an example figure.
You'll notice that the actual figure file is found in the \code{figures} directory.
However, because we have specified this directory in our \verb=\graphicspath= we do not need to explicitly specify the path to the image.

The \code{macros.tex} package also contains some conventional scientific units like \angstrom, \GeV, \Msun, etc. and some editorial tools for highlighting \FIXME{issues}, \CHECK{text to be checked}, \COMMENT{comments}, and \NEW{new additions}.


% ----------------------------------------------------------------------

\section{Methods}
\label{sec:methods}

Similar to the figure before, here we have included a table of data from \code{tables/table.tex}.
Notice that again we are able to reference \tabref{example} with the \verb=\tabref= command using the \code{tab:} prefix.
Also notice that we haven't needed to specify the full path to the table because in the \code{Makefile} we include \code{./tables} directory in the \code{\$TEXINPUTS} environment variable.

\input{table}

Equations appear as follows, and can be referred to as, for example, \eqnref{example} -- just as for tables, we use the \verb=\eqnref= command using the \code{eqn:} prefix.
\begin{equation}
  \label{eqn:example}
  \langle f(k) \rangle = \frac{ \sum_{t=0}^{N}f(t,k) }{N}
\end{equation}


% ----------------------------------------------------------------------

\section{Results}
\label{sec:results}

\figref{example} shows an example figure, referred to with the \verb=\figref= command and the \code{fig:} prefix.

\begin{figure}
\includegraphics[width=0.9\columnwidth]{example.png}
\caption{An example figure: the LSST DESC logo, copied from \code{texmf/logos/desc-logo.png} into \code{figures/example.png}. \label{fig:example}}
\end{figure}


% ----------------------------------------------------------------------

\section{Discussion}
\label{sec:discussion}

If you are planning on committing your paper to GitHub, it's a good idea to write your tex as one sentence per line.
This allows for an easier \code{diff} of changes.
It also makes sense to think of latex as \emph{code}, and sentences as logical statements, occupying one line each.
Each line must ``compile'' in the mind of the reader.


% ----------------------------------------------------------------------

\section{Conclusions}
\label{sec:conclusions}

Here's a summary of what we just reported.

We can draw the following well-organized and neatly-formatted conclusions:
\begin{itemize}
  \item This is important.
  \item We can measure some number with some precision.
  \item This has some implications.
\end{itemize}

Here are some parting thoughts.


% ----------------------------------------------------------------------

\subsection*{Acknowledgments}

Here is where you should add your specific acknowledgments, remembering that some standard thanks will be added via the \code{acknowledgments.tex} and \code{contributions.tex} files.

\input{acknowledgments}

\input{contributions}

%{\it Facilities:} \facility{LSST}

% Include both collaboration papers and external citations:
\bibliography{lsstdesc,main}

\end{document}
% ======================================================================
% 
